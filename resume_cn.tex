\documentclass[a4,12pt]{article}

%%%%%%% --------------------------------------------------------------------------------------
%%%%%%%  STARTING HERE, DO NOT TOUCH ANYTHING 
%%%%%%% --------------------------------------------------------------------------------------
\usepackage[UTF8]{ctex}
\usepackage{latexsym}
\usepackage[empty]{fullpage}
\usepackage{titlesec}
 \usepackage{marvosym}
\usepackage[usenames,dvipsnames]{color}
\usepackage{verbatim}
\usepackage[hidelinks]{hyperref}
\usepackage{fancyhdr}
\usepackage{multicol}
\usepackage{hyperref}
\usepackage{csquotes}
\usepackage{tabularx}
\hypersetup{colorlinks=true,urlcolor=black}
\usepackage[11pt]{moresize}
\usepackage{setspace}
\usepackage{fontspec}
\usepackage[inline]{enumitem}
\usepackage{array}
\usepackage{times}
\newcolumntype{P}[1]{>{\centering\arraybackslash}p{#1}}
\usepackage{anyfontsize}

%%%% Set Margins
\usepackage[margin=1cm, top=1cm]{geometry}

%%%% Set Fonts
% \setmainfont[
% %BoldFont=
% BoldFont=SourceSansPro-Semibold.otf, %SourceSansPro-Bold.otf
% ItalicFont=SourceSansPro-RegularIt.otf
% ]{SourceSansPro-Regular.otf}
% \setsansfont{SourceSansPro-Semibold.otf}
\setmainfont{Times New Roman}


%%%% Set Page
\pagestyle{fancy}
\fancyhf{} 
\fancyfoot{}
\renewcommand{\headrulewidth}{0pt}
\renewcommand{\footrulewidth}{0pt}

%%%% Set URL Style
\urlstyle{same}

%%%% Set Indentation
\raggedbottom
\raggedright
\setlength{\tabcolsep}{0in}

%%%% Set Secondary Color
\definecolor{UI_blue}{RGB}{32, 64, 151}
\definecolor{UI_BLACK}{RGB}{0,0,0}
%%%% Define New Commands
\usepackage[style=nature, maxbibnames=3]{biblatex}
\addbibresource{Publications.bib}

%%%% Bold Name in Publications
\renewcommand*{\mkbibnamegiven}[1]{%
\ifitemannotation{highlight}
{\textbf{#1}}
{#1}}

\renewcommand*{\mkbibnamefamily}[1]{%
\ifitemannotation{highlight}
{\textbf{#1}}
{#1}}

%%%% Set Sections formatting
\titleformat{\section}{
\color{UI_BLACK} \scshape \raggedright \large 
}{}{0em}{}[\vspace{-10pt} \hrulefill \vspace{-6pt}]

%%%% Set Subtext Formatting
\newcommand{\subtext}[1]{
#1\par\vspace{-0.3cm}}

% \newcommand{\subtextit}[1]{\vspace{0.15cm}
% \textit{ #1 \vspace{-0.2cm}} }

%%%% Set Item Spacing
\setlist[itemize]{align=parleft,left=0pt..1em}

%%%% New Itemize "Zitemize" Formatting - tighter spacing than itemize
\newenvironment{zitemize}{
\begin{itemize}\itemsep0pt \parskip0pt \parsep1pt}
{\end{itemize}\vspace{-0.5cm}}


%%%% Define Skills Bold Formatting
\newcommand{\hskills}[1]{
\textbf{\bfseries #1} }

%%%% Set Subsection formatting
\titleformat{\subsection}{\vspace{-0.1cm} 
\bfseries \fontsize{11pt}{2cm}}{}{0em}{}[\vspace{-0.2cm}]

%%%%%%% --------------------------------------------------------------------------------------
%%%%%%% --------------------------------------------------------------------------------------
%%%%%%%  END OF "DO NOT TOUCH" REGION
%%%%%%% --------------------------------------------------------------------------------------
%%%%%%% --------------------------------------------------------------------------------------

\begin{document}

%%%%%%% --------------------------------------------------------------------------------------
%%%%%%%  HEADER
%%%%%%% --------------------------------------------------------------------------------------
\begin{center}
    \begin{minipage}[b]{0.5\textwidth}
            \centering
            {\huge 袁廷洲} \\ %
            \vspace{0.1cm}
            tzy15368@outlook.com;           \href{https://www.linkedin.com/in/tingzhou-yuan-aa7718214;}{linkedin链接}; 
            \href{https://github.com/tzy15368}{github链接}
    \end{minipage}% 
     
\end{center}
\vspace{-0.35cm}
\section{\textbf{教育}}
\subsection*{大连理工大学 \hfill \textbf{Sep 2018 --- Jun 2022}}
\begin{tabular}{p{20em} p{21em} p{43em}}
\textbf{计算机科学与技术} & \textbf{83.4/100}\\
辅修:项目管理 & 85.6/100\\
\end{tabular}

\textbf{\small 相关课程}\\
离散数学;数据结构与算法;编译原理;数据库系统原理…

\begin{tabular}{p{30em} p{21em}}
\textbf{\small 相关奖项} & \\
2020 中国大学生计算机设计大赛 国赛 & 二等奖 \\
2019 IET Present Around the World演讲比赛 大连理工大学赛区 & 二等奖 \\
\end{tabular}

\vspace{-0.3cm}
\subsection*{丹麦技术大学 (DTU) \hfill \textbf{Aug 2021 --- Dec 2021}}
交换生,计算机科学与技术 \\ 
\textbf{\small 相关课程} \\ 
\small 02110 算法2, 02158并发编程, 02269过程挖掘, 02807 数据科学工具\normalsize \\

\vspace{-0.3cm}
\subsection*{伦敦大学学院 (UCL) \hfill \textbf{Jul 2019 --- Aug 2019}}
暑校,统计与概率

\vspace{-0.3cm}
\subsection*{MOOC \hfill \textbf{Jan 2020 --- Dec 2020}}
\begin{tabular}{p{10em} p{21em}}
Coursera.org     & 算法1, 序列模型 \\
MIT     & 6.824分布式系统
\end{tabular}


\vspace{-0.6cm}
\section{\textbf{工作经历}}

\vspace{-0.3cm}
\subsection*{科恩实验室, 腾讯\hfill Jan 2021 --- Mar 2021} 
\subtext{\textit{后端开发实习生}\hfill 上海} 
    \begin{zitemize}
        \item 负责开发GraphQL后端resolver以及对应的SDK接口
        \item 设计、实现了基于AI的二进制文件分析器的Python SDK
        \item 后端文件分析器接入对象存储
        \item 调研并实现了实验性的系统内部防腐层
    \end{zitemize}


%%%%%%% ----------------------------------- Role 5 ----------------------------------- %%%%%%%
\vspace{-0.3cm}
\subsection*{飞书,字节跳动\hfill June 2021 --- Aug 2021} 
\subtext{\textit{暑期实习生} \hfill 北京} 
    \begin{zitemize}
        \item 设计并实现了一套聊天机器人框架、实际应用,用于自动化重复人工操作,实践了RBAC、自动工单处理与敏感资源分配
        \item 编写自动化脚本操作微服务内部CoreDNS、负载均衡配置以协助其他部署调整
        \item 协助SRE同事处理微服务升级hot fix
        \item 聚合k8s监控数据,并针对性地提出资源配置优化方案
        \item 编写物理机磁盘监控agent
    \end{zitemize}

\vspace{-0.3cm}
\section{\textbf{学生工作}}
\subsection*{DUTBIT 工作室 (团委宣传部)\hfill  Oct 2018 --- Jun 2021} 

\subtext{\textit{职责:主任}}
\vspace{-0.1cm}
\begin{zitemize}
    \item 担任新人导师,提供学习方向的指引
    \item 内部技术分享
    \item 建立内部wiki,实现知识积累
\end{zitemize}
\subtext{\textit{职责:前核心开发者}}
\vspace{-0.1cm}
    \begin{zitemize}
        \item 负责设计、开发、迭代团委内部的单体架构多租户社团管理系统。
        \item 使用异步消息队列与多级缓存以显著优化后端性能,将TTFB指标降低了100ms
        \item 接入github actions流水线,实践CI/CD
        \item 集成云原生组件:对象存储,验证码服务,消息队列,负载均衡与CDN,显著降低扩容复杂度
        \item 为校内合作伙伴提供单点登录服务 
    \end{zitemize}
\vspace{-0.3cm}
%%%%%%% ----------------------------------- Role 1 ----------------------------------- %%%%%%%
\section{\textbf{项目}}
\subsection*{Lazarus: {\normalsize\normalfont 轻量级web服务器进程管理平台} \hfill \textbf{Sep 2021 --- Dec 2021}}
\vspace{0.1cm}
\subtext{\textit{独立开发}}
\begin{zitemize}
    \item 设计、实现了可轻松扩容的轻量级主从架构(采用Apache Thrift RPC框架)
    \item 使用基于Open-Resty的方案在七层负载均衡(Nginx)上实现配置热重载和用户鉴权
    \item 使用React.js实现前端管理面板
\end{zitemize}

\vspace{-0.3cm}

\subsection*{LAD全息广告解决方案 \hfill \textbf{Jun 2021 --- Jul 2021}}
\vspace{0.1cm}
\subtext{\textit{项目管理}}
\begin{zitemize}
    \item 项目组内部沟通协调、推动成员间协作
\end{zitemize}

\vspace{0.1cm}
\subtext{\textit{软件开发}}
\begin{zitemize}
    \item 基于树莓派的网络摄像头画面通过websocket数据转发及实时预览和录像
    \item 通过树莓派端前端调用OpenGL渲染和优化后端数据处理链路以及采用SIMD硬件加速实现了4倍的FPS提升
    
\end{zitemize}

\vspace{-0.3cm}
\subsection*{CMM 编译器与运行时环境 \hfill \textbf{Oct 2020 --- Nov 2020}}
\begin{zitemize}
    \item 实现了一个简易的类C语法编译器与对应的虚拟机运行时 
    \item 负责设计并实现了基于栈的虚拟机和对应的PCode解释器
    \item 实现了LL1文法分析器
\end{zitemize}

\vspace{-0.3cm}
\subsection*{md2paper markdown格式转换工具\hfill \textbf{Oct 2019 --- Dec 2019}}
\begin{zitemize}
    \item 自行扩展的markdown语法,实现了对图、表、公式及其引用的完整支持
    \item 采用了基于pyodide的wasm运行时,所有数据只存在前端,部署于\href{https://tzy15368.github.io/md2paper/md2paper.html}{gh-pages}
\end{zitemize}



%%%%%%% --------------------------------------------------------------------------------------
%%%%%%%  SKILLS
%%%%%%% --------------------------------------------------------------------------------------
\section{\textbf{其他技能}}
\begin{tabular}{p{11em} p{1em} p{43em}}
%\hskills{Physics Skills }&  &  A, B, C, D, E, F \\
\hskills{语言} & & GRE 333+5.0, IELTS 7.5, TOEFL 109 \\
\hskills{沟通} & & 双语工作环境 (中英)  \\
\hskills{编程} &  & Python, Golang, JavaScript, Java \\
\hskills{其他经验} & & MySQL, Redis, Influxdb
\end{tabular}
\vspace{-0.2cm}


\end{document}